\documentclass[11pt]{article}

\usepackage[a4paper, margin=2cm]{geometry} % Seitenränder einstellen
\usepackage{setspace} % Paket für Zeilenabstand
\onehalfspacing % 1,5-facher Zeilenabstand
\usepackage{parskip} % Paket für Abstand zwischen Absätzen
\setlength{\parskip}{6pt} % 6 Punkt zusätzlicher Abstand zwischen Absätzen
\usepackage{csquotes} % Paket für Zitate
\usepackage[style=apa,backend=biber]{biblatex} % Paket für Zitate im APA-Stil
\DeclareLanguageMapping{ngerman}{ngerman-apa} % Deutsche Sprache für APA
\addbibresource{literatur.bib} % Literaturdatei einbinden
\usepackage[hidelinks]{hyperref} % Paket für Hyperlinks
\usepackage{amsmath}
\usepackage{amssymb}
\DeclareMathOperator{\arccot}{arccot}
\DeclareMathOperator{\arcoth}{arcoth}
\DeclareMathOperator{\arsinh}{arsinh}
\DeclareMathOperator{\artanh}{artanh}
\DeclareMathOperator{\arcosh}{arcosh}
\DeclareMathOperator{\sgn}{sgn}
\usepackage{multirow}
\usepackage{array}
\usepackage{longtable}
\title{Matrizen}
\begin{document}
\section{Multiplikation von Matrizen}
\underline{Voraussetzung}: \textit{Spaltenzahl} von \textbf{A} stimmt mit \textit{Zeilenzahl} von \textbf{B} ueberein.

\textbf{A} = $( a_{ik})$ seine ein Matrix vm Type $(m, n)$, \textbf{B} = $( b_{ik})$ eine Matrix vom Type $(n,p)$.
\begin{align*}
C &= A \cdot B = \left( c_{ik} \right) \\
\left( c_{ik} \right) &= a_{i1}b_{1k} + a_{i2}b_{2k} + \cdots + a_{in}b_{nk} \\
&= \sum_{j=1}^n{a_{ij}b_{jk}}
\end{align*}
\underline{Beispiel:}
\begin{align*}
C &=
\begin{pmatrix}
1 & 5 \\
2 & 3
\end{pmatrix} \cdot
\begin{pmatrix}
4 & 1 & 2 \\
1 & 0 & 4
\end{pmatrix} =
\begin{pmatrix}
c_{11} & c_{12} & c_{13} \\
c_{21} & c_{22} & c_{23}
\end{pmatrix} \\
c_{11} &= a_{11} b_{11} +  a_{12} b_{21} \\
c_{12} &= a_{11} b_{21} +  a_{12} b_{22} \\
c_{13} &= a_{11} b_{31} +  a_{12} b_{32} \\
\cdots \\
c_{23} &= a_{21} b_{31} +  a_{21} b_{32} \\
\end{align*}


\section{Transponierte Matrix}
Die \textit{Transponierte} $\mathbf{A^T}$ erhaelt man wenn Zeilen und Spalten der Matrix \textbf{A} miteinander vertauscht werden.

\underline{Beispiel:}
\begin{align*}
A &= \begin{pmatrix}
1 & 3 \\
4 & 2 \\
0 & -8
\end{pmatrix} \\
A^T &= \begin{pmatrix}
1 & 4 & 0 \\
3 & 2 & -8
\end{pmatrix}
\end{align*}
\section{inverse Matrix}
Zu jeder \textit{regulaeren} n-reihigen Matrix \textbf{A} gibt es genau eine \textit{inverse} Matrix $\mathbf{A^{-1}}$.

\begin{math}
A^{-1} = \frac{1}{det \mathbf{A}} \cdot 
\begin{pmatrix}
A_{11} & A_{21} & \cdots & A_{n1} \\
A_{12} & A_{22} & \cdots & A_{n2} \\
\cdots & \cdots & & \cdots \\
A_{1n} & A_{2n} & \cdots & A_{nn} \\

\end{pmatrix}
\end{math}
\end{document}