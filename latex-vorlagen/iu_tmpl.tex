\documentclass[11pt]{article}

\usepackage[a4paper, margin=2cm]{geometry} % Seitenränder einstellen
\usepackage{fontspec} % Paket für Schriftart
\setmainfont{Arial} % Schriftart auf Arial setzen
\usepackage{setspace} % Paket für Zeilenabstand
\onehalfspacing % 1,5-facher Zeilenabstand
\usepackage{parskip} % Paket für Abstand zwischen Absätzen
\setlength{\parskip}{6pt} % 6 Punkt zusätzlicher Abstand zwischen Absätzen
\usepackage{csquotes} % Paket für Zitate
\usepackage[style=apa,backend=biber]{biblatex} % Paket für Zitate im APA-Stil
\DeclareLanguageMapping{ngerman}{ngerman-apa} % Deutsche Sprache für APA
\addbibresource{literatur.bib} % Literaturdatei einbinden
\usepackage{tocloft} % Paket für Anpassung des Inhaltsverzeichnisses
\usepackage[hidelinks]{hyperref} % Paket für Hyperlinks

\begin{document}
\pagenumbering{roman}
\begin{titlepage}
    \begin{center}
        \vspace*{2cm}

        {\LARGE \textbf{Titel der Arbeit}}\\[1.5cm] % Titel der Arbeit
        
        {\large \textbf{Advanced Workbook}}\\[1.5cm] % Art der Arbeit

        \textbf{DLBWIRITT01 - Einführung in das wissenschaftliche Arbeiten für IT und Technik}\\[0.5cm] % Kursbezeichnung

        \textbf{Studiengang}\\[1.5cm] % Studiengang

        \begin{tabbing}
            \hspace*{4cm}\=\kill
            Datum: \> \today \\[0.5cm] % Aktuelles Datum
            Name: \> Vorname Nachname\\[0.5cm] % Ihr Name
            Matrikelnummer: \> Matrikelnummer \\[0.5cm] % Ihre Matrikelnummer
        \end{tabbing}

        \vfill

    \end{center}
\end{titlepage}
\tableofcontents
\newpage
\pagenumbering{arabic}
\section{Aufgabe: Wissenschaftstheorie}
\begin{enumerate}
    \item Suche Dir einen wissenschaftlichen Artikel aus dem Fachgebiet der IT und Technik aus (z.B. aus einer
Fachzeitschrift, einem Fachbuch, Sammelband oder Conference Proceedings). Wähle dabei einen Artikel,
der nicht im Skript erwähnt wird. Die Quelle muss die Kriterien für eine gute wissenschaftliche Arbeit
erfüllen, um damit diese und die folgenden Workbook Aufgaben zu bearbeiten (z.B. typische Struktur,
hinreichende Länge und Literaturverzeichnis mit mehreren Quellen). Nutze hierfür bevorzugt die Online-
Bibliothek der Library and Information Services (LIS). Führe den Artikel mit vollständigen
Literaturangaben gemäß den Regelungen aus dem Skript/Zitierleitfaden auf.
    \item Fasse
        \begin{enumerate}
            \item die Problemstellung/den Hintergrund des Artikels,
            \item die Zielsetzung/Forschungsfrage(n)/Ziele und
            \item die wesentlichen Ergebnisse und Schlussfolgerungen aus dem Artikel in eigenen Worten zusammen.
Das heißt, Du sollst den Text indirekt zitieren, gemäß den Zitationsregeln aus dem Skript und
Zitierleitfaden.
        \end{enumerate}
    \item Bestimme die Forschungsmethodik aus dem Artikel und nenne drei Argumente, warum es sich um eine
qualitative bzw. quantitative Forschungsmethodik oder bspw. um eine Literatur- oder Übersichtsarbeit
handelt.
\end{enumerate}
\printbibliography

\end{document}
